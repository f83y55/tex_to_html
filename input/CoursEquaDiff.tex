\documentclass[a4paper,11pt]{article}
\setlength{\columnseprule}{1pt}
\usepackage[utf8]{inputenc}
\usepackage[francais]{babel}
\usepackage[a4paper]{geometry}
\geometry{verbose,tmargin=1cm,bmargin=1cm,lmargin=1cm,rmargin=1cm,headheight=5mm,headsep=5mm,footskip=5mm}

\usepackage[pdftex]{graphicx}
\usepackage{amsmath}
\usepackage{graphics}
\usepackage{latexsym}
\usepackage{eurosym}
\usepackage{multicol}
\usepackage{array}
\usepackage{variations}
\usepackage{cancel}
\usepackage{enumerate}
\usepackage{xcolor}
\usepackage{titlesec}


\definecolor{colortitle1}{RGB}{0,50,50}
\definecolor{colortitle2}{RGB}{120,0,0}

\definecolor{colorsection}{RGB}{120,0,0}
\titleformat{\section}
{\color{colorsection}\normalfont\Large\bfseries}
{\color{colorsection}\thesection}{1em}{}

\definecolor{colorsubsection}{RGB}{0,90,0}
\titleformat{\subsection}
{\color{colorsubsection}\normalfont\large\bfseries}
{\color{colorsubsection}\thesubsection}{1em}{}

\renewcommand{\FrenchLabelItem}{$\bullet$}

\font\ensemble=msbm10
\def\N{\hbox{\ensemble{N}}}
\def\Z{\hbox{\ensemble{Z}}}
\def\Q{\hbox{\ensemble{Q}}}
\def\R{\hbox{\ensemble{R}}}
\def\F{\hbox{\ensemble{F}}}
\def\D{\hbox{\ensemble{D}}}
\def\e{\textrm{e}}

\newcounter{numdefi} \setcounter{numdefi}{1}
\def\defi{\textbf{Définition \thenumdefi~: \addtocounter{numdefi}{1} }}
\newcounter{numprop} \setcounter{numprop}{1}
\def\prop{\textbf{Propriété \thenumprop~: \addtocounter{numprop}{1} }}
\newcounter{numexem} \setcounter{numexem}{1}
\def\exem{\textbf{Exemple \thenumexem~: \addtocounter{numexem}{1} }}
\newcounter{numlemm} \setcounter{numlemm}{1}
\def\lemm{\textbf{Lemme \thenumlemm~: \addtocounter{numlemm}{1} }}
\newcounter{numcoro} \setcounter{numcoro}{1}
\def\coro{\textbf{Corollaire \thenumcoro~: \addtocounter{numcoro}{1} }}
\newcounter{numrema} \setcounter{numrema}{1}
\def\rema{\textbf{Remarque \thenumrema~: \addtocounter{numrema}{1} }}
\newcounter{numdemo} \setcounter{numdemo}{1}
\def\demo{\textbf{Démonstration \thenumdemo~: \addtocounter{numdemo}{1} }}
\newcounter{nummeth} \setcounter{nummeth}{1}
\def\meth{{\color{colorsubsection}\textbf{Méthode \thenummeth~: \addtocounter{nummeth}{1} }}}
\newcounter{numexo} \setcounter{numexo}{1}
\definecolor{colorexo}{RGB}{70,0,90}
\def\exo{{\color{colorexo}\textbf{Exercice \thenumexo~: \addtocounter{numexo}{1} }}}



\parskip=2mm
\date{}
\author{}
\title{}
\begin{document}

\begin{center} \Huge{\color{colortitle1}\fbox{\textbf{Équations Différentielles}}}\normalsize \end{center}
\begin{multicols}{2}
\section{Généralités}
\subsection{Notion d'équation différentielle}

\exo \textbf{Notions de mécanique} 
\begin{enumerate}[a)]
\item On sait qu'un objet en chute libre pour lequel les frottements à l'air sont négligeables chute avec une accélération constante, égale, sur Terre, à $g=9,\!81\approx10$\,m/s$^2$.\\
\emph{Si l'objet est lâché sans vitesse initiale, quelle est sa vitesse au bout d'une seconde ? De deux ? De $10$\,s ?}
\item \emph{Même question si on le lance vers le bas avec une vitesse de $5$\,m/s.}
\item \textbf{Rappel : } La dérivée (par rapport au temps $t$) de la position $x(t)$ est la vitesse (instantanée) $v(t)$. L'accélération $a(t)$ est la dérivée de la vitesse par rapport à $t$, et donc aussi la dérivée seconde de la position par rapport au temps.\\
 On a $a(t)=10$ ; \emph{Donner $v(t)$ et $x(t)$, sachant que $v(0)=5$m/s et $x(0)=300$m.}
\end{enumerate}

\rema Si l'objet est soumis à des frottements (linéaires : proportionnels à la vitesse acquise par l'objet durant sa chute) de coefficients $K$, sa vitesse vérifie alors $a(t)+Kv(t)=g$ ; le problème étant que comme $a(t)=v'(t)$, on obtient :
$$v'(t)+Kv(t)=g$$
Il semble alors difficile de déterminer $v(t)$... C'est pourtant réalisable, et c'est le but de ce cours.
\begin{center}
\includegraphics[width=9cm]{chuteLibre.png}
\end{center}
\rema $v_{\lim}$ est la vélocité terminale : les frottements conduisent à l'apparition d'une vitesse limite lors de la chute.
\subsection{Définition}

\defi \emph{Une \underline{équation différentielle} est une équation dont l'inconnue est une fonction $y$ de la variable $t$ ; l'équation est dite différentielle car elle fait apparaître un lien entre les dérivées de $y$.}
\begin{itemize}
\item \textbf{1er ordre : } $$(E_1)~:~~ay'+by=d(t)$$
\item \textbf{2nd ordre : } $$(E_2)~:~~ay''+by'+cy=d(t)$$
\end{itemize}
avec $a\neq0$, $b$, $c$ constantes réelles et $d$ fonction de $t$.\par


\subsection{Notion de solution (particulière)}
\defi \emph{Toute fonction vérifiant l'équation différentielle $(E)$ est appelée \underline{solution particulière} de $(E)$.}\par

\rema Une équation différentielle a une infinité de solutions (particulières).\par

\exo \emph{Vérifier que $f(t)=\e^{-2x}+4$ est une solution particulière de $(E)~:~~y'+2y=8$.}\par

{\color{colorsubsection} \meth \begin{enumerate}[1)]
\item On calcule $f'(t)$ (et $f''(t)$ dans le cas d'une équation du second ordre) ;
\item on réinjecte $f(t)$ et ses dérivées dans l'équation ;
\item on simplifie pour vérifier que l'égalité de l'équation est bien vérifiée.\\
\end{enumerate}
}
\exo \emph{Vérifier que $f(t)=\e^{-t}+3$ et $g(t)=t\e^{-t}+3$ sont deux solutions particulières de $(E)~:~~y''+2y'+y=3$.}\par

\rema Une équation différentielle admet la plupart du temps au moins une solution particulière $f(t)$ de même forme que son second membre ; ainsi, si le second membre est une constante, on pourra poser $f(t)=A$ ($A$ cte); si le second membre est un polynôme de degré $2$, on poser $f(t)=At^2+Bt+C$ ($A,B,C$ ctes) ; si le second membre est $-10\e^{-2t}$, on posera $f(t)=Ae^{-2t}$ ($A$ cte)... La plupart du temps, la forme de $f(t)$ est directement donnée par l'énoncé du problème.\par

{\color{colorsubsection} \meth \textbf{Trouver une solution particulière }$f(t)$
\begin{enumerate}[a)]
\item On pose $f(t)=$ même forme que le second membre, avec des constantes $A$, $B$,... ;
\item on calcule $f'(t)$ (et $f''(t)$ si besoin) (qui vont dépendre des constantes utilisées) ;
\item on réinjecte $f(t)$ et ses dérivées dans l'équation ;
\item on simplifie ; 
\item on détermine les conditions sur les constantes pour que l'égalité de l'équation soit bien vérifiée ;
\item on remplace les constantes par les valeurs obtenues à l'étape précédente pour écrire l'expression définitive de $f(t)$.
\end{enumerate}
}

\exo \emph{Trouver une solution particulière de $y'-3y=1-t$ sous la forme $f(t)=At+B$}


\subsection{Notion de solution générale}
\rema Les équations différentielles ont une infinité de solutions ; pourtant, celles-ci ne diffèrent que d'une ou deux constantes multiplicatives. Il est ainsi possible d'écrire une formule donnant la forme de toutes les solutions d'une équation différentielle.\par
\defi \emph{On appelle cette formule «\underline{solution générale}» de l'équation différentielle.}\par
\exem $f(t)=k\e^{-2x}+4$, où $k$ est une constante pouvant prendre n'importe quelle valeur réelle, est la solution générale de $(E)~:~~y'+2y=8$.\par
\rema Dans la solution générale, affecter une valeur réelle quelconque à $k$ permet d'obtenir une solution particulière de l'équation différentielle étudiée.\\
Il existe une méthode permettant d'obtenir facilement ces solutions générales.

\section{Méthode de résolution}

\subsection{Équation homogène}
\defi \emph{Pour transformer un équation différentielle en son \underline{équation homogène} associée (on dit aussi «sans second membre», celui-ci devant être égal à $0$), on ne conserve que les termes contenant $y$, $y'$ ou $y''$ (avec leurs facteurs multiplicatifs).}\par
\noindent\textbf{1er ordre : } $(E_1^*)~:~~ay'+by=0$\\
\textbf{2nd ordre : } $(E_2^*)~:~~ay''+by'+cy=0$ \par
\rema L'équation homogène associée est notée $(E^*)$ pour permettre de garder les indices s'il y a plusieurs équations, sinon notée $(E_0)$ parfois.
\subsection{Équation caractéristique}
\defi\emph{C'est une équation numérique obtenue en remplaçant dans $(E^*)$ les symboles $y$, $y'$ et $y''$ par $1$, $r$ et $r^2$ : la puissance de $r$ correspond à l'ordre de dérivation.}\\
\textbf{1er ordre : } $(E_1^c)~:~~ar+b=0$\\
\textbf{2nd ordre : } $(E_2^c)~:~~ar^2+br+c=0$ \par

\exo \emph{Donner les équations homogènes et caractéristiques des équations différentielles suivantes : }
\begin{enumerate}[a)]
\item $(E_1)~:~~2y'+10y=10 $
\item $(E_2)~:~~2y'+10y=2\e^{-2t}$
\item $(E_3)~:~~y''+4y'+8y=4$\\
\end{enumerate}

\prop \textbf{Solutions générales d'une équation \underline{homogène}}
\begin{itemize} 
\item \textbf{1er ordre : } $(E^c)~:~~$ une solution $r$, solution générale de $(E_1^*)$ : \fbox{$k\e^{rt}$}.\\
\item \textbf{2nd ordre : } $(E^c)~:~~ar^2+br+c=0$ ; 3 cas :
\begin{itemize}
\item $\Delta>0$ : $(E^c)$ a deux solutions $r_1$ et $r_2$ ; solution générale de $(E_2^*)$ : \fbox{$k\e^{r_1t}+l\e^{r_2t}$}.\\
\item $\Delta=0$ : $(E^c)$ a une solution $r_0$ ; solution générale de $(E_2^*)$ : \fbox{$k\e^{r_0t}+lt\e^{r_0t}$}.\\
\item $\Delta<0$ : $(E^c)$ a deux solutions $r_1=\alpha-i\beta$ et $r_2=\alpha+i\beta$, complexes conjuguées (on choisit $\beta>0$) ; solution générale de $(E_2^*)$ : \fbox{$k\e^{\alpha t}\cos(\beta t)+l\e^{\alpha t}\sin(\beta t)$}.\\
\end{itemize}
\end{itemize}

\exo \emph{Donner les solutions générales des équations différentielles homogènes suivantes : }
\begin{enumerate}[a)]
\item $(E_1^*)~:~~2y'+10y=0 $
\item $(E_2^*)~:~~y''+\omega^2y=0$ ($\omega$ cte)
\item $(E_3^*)~:~~y''+4y'+8y=0$
\end{enumerate}


\subsection{Résoudre une équation différentielle $(E)$ quelconque}
\prop On utilise le \underline{principe de superposition} :\par
{\color{colorsubsection} \meth \begin{enumerate}[1)]
\item On recherche une solution particulière $f(t)$ de $(E)$ ;
\item on écrit $(E^*)$, l'équation homogène associée à $(E)$ ;
\item on en déduit l'équation caractéristique $(E^c)$, que l'on résout ;
\item on en déduit les solutions générales de $(E^*)$ ;
\item On obtient les solutions générales de $(E)$ en ajoutant à une solution particulière de $(E)$, comme ici $f(t)$, les solutions générales de $(E^*)$.\\
\end{enumerate}
}
\exo \emph{Donner les solutions générales des équations différentielles suivantes : }
\begin{enumerate}[a)]
\item $(E_1)~:~~2y'+10y=10 $
\item $(E_2)~:~~2y'+10y=2\e^{-2t}$
\item $(E_3)~:~~y''+4y'+8y=4$\\
\end{enumerate}

\section{Conditions initiales/particulières}
\defi \emph{Poser une \underline{condition initiale} revient à déterminer quelle solution particulière de l'équation différentielle vaut une valeur donnée $y(0)$ à $t=0$ ($y'(0)$ est aussi donnée dans le cas du 2nd ordre). Si on se place en un autre instant $t>0$, on parle de \underline{condition particulière}.}\par
{\color{colorsubsection} \meth
On recherche une (autre) solution particulière $\phi$ ou $g$ de $(E)$ en remplaçant dans la formule donnant les solutions générales de $(E)$ (2nd ordre : et dans sa dérivée) $t$ par la valeur donnée et en résolvant l'équation en $k$ (2nd ordre : le système en $k$ et $l$) obtenu(e).
}

\exo \begin{enumerate}[a)]
\item \emph{Après avoir recherché une solution particulière de $(E_1)$ sous la forme $f(t)=A$ constante, résoudre :}\\
$$\left\{\begin{array}{l}
(E_1)~:~~2y'+10y=10 \\
~~~~~~~~~y(0)=4\\
\end{array}\right.$$
\item \emph{Après avoir recherché une solution particulière de $(E_2)$ sous la forme $f(t)=Ae ^{-2t}$, résoudre :}\\
$$\left\{\begin{array}{l}
(E_2)~:~~2y'+10y=2\e^{-2t} \\
~~~~~~~~~y(0)=4\\
\end{array}\right.$$
\item \emph{Après avoir recherché une solution particulière de $(E_3)$ sous la forme $f(t)=A$ constante, résoudre :}\\
$$\left\{\begin{array}{l}
(E_3)~:~~y''+4y'+8y=4 \\
~~~~~~~~~y(0)=2\textrm{~et~} y'(0)=-1\\
\end{array}\right.$$
\end{enumerate}

\section{Geogebra}

\subsection{ChampVecteurs}
\texttt{ChampVecteurs[ <f(x,y)> ]}\\
    Représente un champ de vecteurs pour l'équation différentielle $\mathrm{\mathsf{\frac{dy}{dx}=f(x,y)}}$\\
    \exem \texttt{ChampVecteurs[x+y]} représente le champ de vecteurs pour $y'(x)=x+y(x)$

\subsection{RésolEquaDiff}
\textbf{Résolution numérique :}\\
\texttt{RésolEquaDiff[ <f(x,y)>, <x initial>, <y initial>, <x final>, <pas> ] }
    résout numériquement une équation différentielle d'ordre un : $\mathrm{\mathsf{\frac{dy}{dx}=f(x,y)}}$
à partir d'un point donné par ses coordonnées, avec un pas donné.\par

\textbf{Résolution formelle :}\\
\texttt{RésolEquaDiff[ <f(x, y)> ]}\\
    essaye de trouver la solution exacte de l'équation différentielle d'ordre un $\mathrm{\mathsf{\frac{dy}{dx}(x)=f(x,y(x))}}$\par
\exo \emph{Reprendre l'exercice précédent à l'aide de Geogebra.}
 
\section{Problèmes}
\exo L'allongement $x$ d'un ressort suit l'équation $(E)$ :
$$(E)~:~x''(t)+2x'(t)+(1+64\pi^2)x(t)=0$$
On lâche le ressort ($x'(0)=0$) avec un allongement de $x(0)=0,\!1$m.\\
\emph{Déterminer l'allongement du ressort au cours du temps.}\par
\exo Un parachutiste saute d'une hauteur $x(0)=8\,500$\,m sans vitesse initiale et ouvre son parachute au bout de $120$ s. Son poids $\overrightarrow{P}=m\overrightarrow{g}$ est conditionné par l'accélération de pesanteur $\overrightarrow{g}$, vers le bas, de norme $10$m/s$^2$. On considère que le parachutiste subit les frottements linéaires de l'air de coefficients $K_1=0,\!15$ lors de la chute libre puis $K_2=10$ après l'ouverture du parachute.\\
Le principe fondamental de la dynamique Newtonienne conduit à :
$$(E)~:~~v'+K_{1|2}v=g$$

\emph{Résoudre l'équation différentielle $(E)$ et tracer les courbes de position, vitesse et accélération du parachutiste.}\par

\exo \emph{Étudier l'équation différentielle vérifiée par un circuit série RLC.}

\end{multicols}




\end{document}
