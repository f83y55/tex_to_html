\documentclass[a4paper,11pt]{article} \usepackage{FBarticle} \mapage{831}{Probabilités 2} % document papier
%\documentclass[9pt]{beamer}  \usepackage{FBbeamer} % \mapage{704}{Fonctions exponentielles} % présentation

\begin{document}
\titre{Probabilités 2}{Cours}{ -- Lois continues}

\begin{multicols}{2}

\section{Généralités}
On sait qu'un phénomène doit se produire à un instant donné, entre $0$h et $10$h. Quelle est la probabilité qu'il ait lieu entre $7$h et $8$h ?\\
Ce type de problème est modélisé en utilisant des lois de probabilité continues.

\subsection{Notion de densité}
Soit $I$ un intervalle et $f$ une fonction définie sur $I$.\par
\defi{$f$ est appelée \underline{densité sur $I$} lorsque :}

\begin{itemize}
	\item $f\geq0$ ;
	\item $\displaystyle\int_I\,f(t)\,\textrm{d}t=1$.
\end{itemize}
\exem $f(t)=\dfrac{3}{500}x(10-x)$ est une densité sur $[0;10]$ :\\
\includegraphics[width=7cm]{dens1.png}\par

\exo{Déterminer la constante $A$ pour que $g(x)=Ax$ soit une densité sur $[0;10]$.}



\subsection{Fonction de répartition}
Soit $I$ un intervalle et $f$ une densité sur $I$.\\
On pose $a=\inf(I)$  et $b=\sup(I)$ (on peut avoir $a=-\infty$ et $b=+\infty$) et :

$$F(t)=\int_a^t\,f(u)\,\textrm{d}u$$

($F$ est ainsi la primitive de $f$ qui s'annule en $t=a$)\par

\defi{Cette primitive $F$ est appelée \underline{fonction de répartition}.}\par

\prop \begin{itemize}
	\item $F$ est croissante ;
	\item $F(a^+)=0$ et $F(b^-)=1$.
\end{itemize}

\exem \\
\includegraphics[width=7cm]{dens2.png}

\subsection{Loi de probabilité admettant une densité}
Soit $c<d$ deux nombres de l'intervalle $I$.\\
\defi{Si $X$ est une variable aléatoire qui suit une loi de densité $f$, alors on a :}

$$P(c\leq X\leq d)=\int_c^d\,f(t)\,\textrm{d}t=F(d)-F(c)$$
\exem $c=4$ et $d=6,\!5$\\
\includegraphics[width=7cm]{dens3.png}\par

\rema Dans le cas d'une variable aléatoire $X$ suivant une loi provenant d'une densité, 
\begin{itemize}
\item  pour n'importe quelle valeur de $c$ dans $I$,\\ $P(X=c)=0$ ;
\item $P(X<d)\,=\,P(X\leq d)$ ;
\item $P(c<X)\,=\,P(c\leq X)$ ;
\item de même, \\
	$P(c<X<d)=P(c\leq X <d)=P(c<X\leq d)=P(c\leq X\leq d)$.
\end{itemize}

\subsection{Espérance et variance}
Soit $X$ une variable aléatoire provenant d'une densité $f$ définie sur un intervalle $I$.\par

\defi{Espérance :}\\
$$\mu=E(X)=\int_I\,xf(x)\,\textrm{d}x$$

\prop$X$ et $Y$ sont deux variables aléatoires, $a$ et $b$ deux constantes.
 \begin{itemize}
\item $E(aX+b)=aE(X)+b$
\item $E(aX+bY)=aE(X)+bE(Y)$ (linéarité)
\item Si $X$ et $Y$ sont \textbf{indépendantes} :\\
$E(XY)=E(X)E(Y)$\\
\end{itemize}

\defi{Variance, Écart-type :}\\
$$V(X)=\int_I\,\left[x-E(X)\right]^2f(x)\,\textrm{d}x~~~\textrm{et}~~~\sigma(X)=\sqrt{V(X)}$$

\prop$X$ et $Y$ sont deux variables aléatoires, $a$ et $b$ deux constantes.
 \begin{itemize}
\item $V(aX+b)=a^2V(X)$ et $\sigma(aX+b)=|a|\sigma(X)$ ;
\item Si $X$ et $Y$ sont \textbf{indépendantes} :\\
$V(aX+bY)=a^2V(X)+b^2V(Y)$ \\ et $\sigma(aX+bY)=\sqrt{[a\sigma(X)]^2+[b\sigma(Y)]^2}$
\end{itemize}


\section{Exemples de lois continues}
\subsection{Loi uniforme}
Cette loi existe sur tout intervalle $I$ borné (c'est à dire qu'aucune des bornes de $I$ ne doit être infinie). On sait qu'un événement va se produire à un instant dans l'intervalle de temps $I=[a;b]$, mais on ne sait pas quand et il n'y a aucune raison de privilégier telle ou telle date.\par

\defi{\textbf{Loi uniforme} sur $I=[a;b]$ : }
\emph{Cette loi a pour densité la fonction constante sur $I$ }:
$$f(x)=\dfrac{1}{b-a}$$
\exem Densité uniforme sur $[2;5,5]$ :\\
\includegraphics[width=8.5cm]{dens4.png}

\prop \textbf{Calcul effectif :}\\
Pour $c<d$ deux constantes dans $[a;b]$ :\\
$$P(c<X<d)=\dfrac{d-c}{b-a}$$

\exo{Le démontrer.} %TS

\prop \textbf{Espérance, variance, écart-type :}\\
Si $X$ suit une loi uniforme sur $[a;b]$ :
\begin{itemize}
\item $E(X)=\dfrac{a+b}{2}$ (milieu de $[a;b]$) ;
\item $V(X)=\dfrac{(b-a)^2}{12}$ et $\sigma(X)=\dfrac{b-a}{2\sqrt{3}}$.\\
\end{itemize}

\exo{Le démontrer.} %TS

\exo{Mike peut peut débarquer n'importe quand entre $12\,h$ et $14\,h$. \\
\begin{enumerate}[a)]
\item Quelle est la probabilité qu'il débarque entre $13\,h$ et $13\,h\,30$ ?
\item À quelle heure, en moyenne, débarque-t-il ?
\item Il est $13\,h$ et Mike se fait toujours attendre. Sachant qu'il n'est toujours pas là à $13\,h$, quelle est la probabilité qu'il arrive avant $13\,h\,30$ ?
\end{enumerate}
}

\vspace{5cm}


\subsection{Loi exponentielle}


Cette loi existe sur $I=\left[0;+\infty\right[$ ; elle modélise la durée de vie d'un phénomène \emph{sans mémoire}, comme la durée de vie d'un composant électrique.\par
\defi{ \textbf{Loi exponentielle de paramètre $\lambda$ : }\\
Cette loi a pour densité la fonction  : }
$$f(x)=\lambda\textrm{e}^{-\lambda x}$$
\exem Densité exponentielle de paramètre $1/4$ :\\
\includegraphics[width=8.5cm]{dens5.png}

\prop \textbf{Calcul effectif :}\\
Pour $c$ constante positive : \\
$$P(X<c)=\int_0^c \lambda\textrm{e}^{-\lambda x}\,\textrm{d}x$$

\exo{
Le composant $A$ a une durée de vie qui suit une loi exponentielle de paramètre $0,0004$\\
Calculer la probabilité, arrondie à $10^{-2}$, que le composant $A$ ait une durée de vie strictement inférieure à $1\,000$ heures.}

\prop \textbf{Espérance, variance, écart-type :}\\
Si $X$ suit une loi exponentielle de paramètre $\lambda$ :
$$E(X)=\dfrac{1}{\lambda}~~~;~~~V(X)=\dfrac{1}{\lambda^2}~~~;~~~\sigma(X)=\dfrac{1}{\lambda}$$

\exo{Démontrer que $F(x)=-\left(x+\frac{1}{\lambda}\right)\e^{-\lambda x}$ est une primitive de $\lambda x\e^{-\lambda x}$ ; démontrer alors la formule donnant l'espérance d'une loi exponentielle.} %TS

\prop \textbf{Loi sans mémoire :}\\
La loi exponentielle est dite «sans mémoire», dans le sens où si $c>a>0$ sont donnés, alors :
$$P_{X>a}(X<c)=P(x<c-a)$$


\exo{Le composant $A$ a une durée de vie qui suit une loi exponentielle de paramètre $0,0004$.\\
Sachant que le composant $A$ a tenu déjà $1000$ heures, calculer la probabilité, arrondie à $10^{-2}$, que le composant $A$ ait une durée de vie strictement inférieure à $2\,000$ heures.}


\section{Loi normale}

\subsection{Premières propriétés}

Cette loi est définie sur $\R=\left]-\infty;+\infty\right[$ ; elle est très fréquemment observée dans l'étude de phénomènes physiques ou biologiques.\par
\defi{\textbf{Loi normale d'espérance $\mu$ et d'écart-type $\sigma$ : }\\
Cette loi a pour densité la fonction  : }\\
$$f(x)=\dfrac{1}{\sigma\sqrt{2\pi}}\textrm{e}^{-\frac{1}{2}\left(\frac{x-\mu}{\sigma}\right)^2}$$

\exem Densité normale avec $\mu=2$ et $\sigma=\dfrac{1}{2}$ :\\
\includegraphics[width=8.5cm]{dens6.png}\par

\prop La courbe de $f$ est symétrique par rapport à la droite verticale d'équation $x=\mu$ ; plus $\sigma$ est grand, plus la bosse autour de $\mu$ est applatie ; l'aire sous cette courbe sur $\R$ reste toujours $1$.\par

\defi{ \begin{itemize}
\item Si $\mu=0$, on dit que la loi normale est \underline{centrée}.
\item Si $\sigma=1$, on dit que la loi normale est \underline{réduite}.
\item Si $\mu=0$ et $\sigma=1$, on dit que la loi normale est \underline{centrée-réduite}.
\end{itemize} }

\prop Si $X\sim\mathcal{N}(\mu;\sigma^2)$, alors $Z=\dfrac{X-\mu}{\sigma}$ suit une loi normale centrée réduite.\par


\prop Il est important de mémoriser les 3 intervalles suivants :\\ 
Si $X\sim\mathcal{N}(\mu;\sigma^2)$, alors :  %TS
\begin{itemize} 
\item $P(\mu-\sigma\leq X\leq \mu+\sigma)\approx68\%$ (à $10^{-2}$) ;
\item $P(\mu-2\sigma\leq X\leq\mu+2\sigma)\approx95\%$ (à $10^{-2}$) ;
\item $P(\mu-3\sigma\leq X\leq \mu+3\sigma)\approx99,7\%$ (à $10^{-3}$).\\
\end{itemize}
\defi{On appelle l'intervalle $\left[\mu-2\sigma;\mu+2\sigma\right]$ : \underline{plage de normalité}.}\par

\noindent\includegraphics[width=9.5cm]{distnorm.png}\par

\exo{Déterminer $P(\mu\leq X\leq \mu+3\sigma)$.}\par

\rema La densité d'une loi normale n'admettant pas de primitive «commune», on utilise la calculatrice ou des tables pour calculer les probabilités d'une variable aléatoire qui suit une loi normale.\par

\rema \textbf{Calculatrice :}\\
Si $X\sim\mathcal{N}(\mu,\sigma)$ :
\begin{itemize}
\item TI : dans \fbox{Distrib}(2nde-Var) : \\
$P(a\leq X\leq b)=\,$normalFrep($a,b,\mu,\sigma$) \\
\item Casio : \fbox{OPTN} / STATS / DIST / Norm \\
$P(a\leq X\leq b)=\,$Ncd($a,b,\sigma,\mu$)\\
penser à mettre en mode «variable» ($\neq$ liste).
\end{itemize}


\rema Pour $a=-\infty$, prendre $a=-10^{99}$ ou pour $b=+\infty$, prendre $b=10^{99}$.

\subsection{Coefficients $u_\alpha$ ; inversion de la loi normale}

\prop Si $X$ suit une loi normale de moyenne $\mu$ et d'écart-type $\sigma$, alors pour tout $0<\alpha\leq1$, il existe un unique $u_\alpha\geq0$ tel que :
$$P(\mu-u_\alpha \sigma \leq X \leq \mu+u_\alpha \sigma)=1-\alpha$$

\demo \exo{On pose pour $t\geq0$ : $\phi(t)=P(\mu-t \sigma \leq X \leq \mu+t \sigma)$
\begin{enumerate}[a)]
\item Calculer $\phi(0)$ et $\displaystyle\lim_{+\infty}\phi$.
\item Justifier que $\phi$ est continue et strictement croissante.
\item Conclure.
\end{enumerate}
}

\exo{Démontrer que :
$$\begin{array}{l}
~~~~P(\mu-u_\alpha \sigma \leq X \leq \mu+u_\alpha \sigma)=1-\alpha \\
\Leftrightarrow~~P( X \leq \mu+u_\alpha \sigma)=1-\dfrac{\alpha}{2} \\
\end{array}$$
}

\rema \textbf{Calculatrice :}\\
Calcul de $a$ tel que $P(X\leq a)=p$ (avec $0<p<1$ donné) :
\begin{itemize}
\item TI : dans \fbox{Distrib}(2nde-Var) : \\
$\textrm{invNorm}(p,\mu,\sigma)$ ou $\textrm{FracNormale}(p,\mu,\sigma)$ \\
\item Casio : \fbox{OPTN} / STATS / DIST / Norm \\
$\textrm{invN}(p,\sigma,\mu$)\\
\end{itemize}


\exo{Compléter le tableau suivant (à $10^{-2}$) :} \\
\hspace*{-4mm}\begin{tabular}{|l||c|c|c|c|c|c|}\hline
Risque $\alpha$   & $0,1\%$ & $0,5\%$ & $~~1\%$ & $~~3\%$ & $~~5\%$ & $10\%$ \\ \hline
Seuil  $1-\alpha$ &         &         &         &         &         &        \\ \hline
$u_\alpha$        &         &         &         &         &         &        \\ \hline \hline
Risque $\alpha$   & $20\%$ & $30\%$ & $ 40\%$ & $ 50\%$ & $ 75\%$ & $ 90\%$ \\ \hline
Seuil  $1-\alpha$ &         &         &         &         &         &        \\ \hline
$u_\alpha$        &         &         &         &         &         &        \\ \hline
\end{tabular}

\subsection{Exercices}
\exo{On admet que le temps passé en heures chaque jour devant la TV peut être modélisé par une variable aléatoire $X$ suivant une loi normale de moyenne $4\,h$ et d'écart-type $45\,$min.
On donnera les résultats à $10^{-3}$.
\begin{enumerate}[a)]
\item Déterminer le pourcentage de personnes regardant la télévision entre $3$ et $5$ heures par jour.
\item Déterminer le pourcentage de personnes regardant la télévision moins de $2$ heures par jour
\item Déterminer les trois nombres $Q_1$, $Med$ et $Q_3$ tels que :
$$P(x<Q_1)=\dfrac{1}{4}~;~P(x<Med)=\dfrac{1}{2}~;~P(X<Q_3)=\dfrac{3}{4}$$
\end{enumerate}
}

\vspace{3cm}



\exo{Dans une population, le résultat $X$ au test du QI d'une personne prise au hasard suit une loi normale de moyenne $100$ et d'écart-type $15$.
\begin{enumerate}[a)]
\item Déterminer le pourcentage de personnes ayant un QI supérieur à $90$ ; inférieur à $85$ ; entre $70$ et $90$.
\item Déterminer le réel $k$ tel que $P(X<k)=0,9$ ; interpréter.
\item Déterminer la valeur du réel $l$ tel que $60\%$ des personnes ont un QI supérieur à $l$.
\end{enumerate}
}




\subsection{Approximation d'une loi binomiale par une loi normale}

\prop Si $X\sim\mathcal{B}(n,p)$, avec $n$ assez grand (en pratique on prend $np(1-p)>9$), alors la loi de $X$ est proche de celle de la :loi normale $\mathcal{N}(\mu,\sigma)$ de même espérance (prendre $\mu=np$) et de même écart type ( prendre $\sigma=\sqrt{np(1-p)}$).\par

\textbf{Problème : } Pour une loi normale, la probabilité d’une valeur isolée est nulle. Il semble donc impossible de
calculer $P(X = k)$ avec cette approximation. \\
Approcher la loi binomiale par la loi normale c’est remplacer une loi discrète (celle de $X\sim\mathcal{B}(n,p)$) par une loi
continue (celle de $X_c\sim\mathcal{N}(\mu=np,\sigma=\sqrt{np(1-p)})$).\par
\textbf{Solution : }On remplace donc la probabilité de la valeur isolée $x$ de la
variable $X$ par celle d’un intervalle de longueur $1$ centré en $x$
pour la variable $X_c$ :\\
$P(X = x) \approx P(x -0,5 < X_c < x + 0,5)$\par
\defi{Cette opération s’appelle la \underline{correction de continuité}.}\par
\prop La variable discrète $X$ étant approchée par la variable continue $X_c$ , on utilise les règles suivantes d’approximation :
\begin{itemize}
\item $P(X < n)$ s’obtient avec $P(X_c < n – 0,5)$ ;
\item $P(X \leq n)$ s’obtient avec $P(X_c < n + 0,5)$ ;
\item $P(X > n)$ s’obtient avec $P(X_c > n + 0,5)$ ;
\item $P(X \geq n)$ s’obtient avec $P(X_c > n – 0,5)$.
\end{itemize}
On calcule par exemple $P(a < X \leq b)$ \\ avec $P(a + 0,5 < X_c < b + 0,5)$.\par
\exem \par
Soit $X\sim\mathcal{B}\left(50, \frac{1}{2}\right)$.\\
Les conditions d'approximations de la loi de $X$ 
par une loi normale sont remplies, et l'on peut considérer que $X$ suit à peu près la loi $\mathcal{N}\left(25, \frac{25}{2}\right)$.\\
Évaluons alors de deux façons $P(24 \leq X \leq 26)$ :
\begin{itemize}
\item En valeur exacte avec la loi binomiale :\\
$P(X = 24) + P(X = 25) + P(X = 26) \approx 0,3282$ 
\item En valeur approchée avec la loi normale :\\
$P(24 \leq X \leq 26) \approx 0,2222$
\item En valeur approchée avec la loi normale corrigée par continuité :\\
$P(23,5 \leq X \leq 26,5) \approx 0,3286$\\
Le résultat est bien meilleur en tenant compte de la correction par continuité.
\end{itemize}

\section{Théorème central-limite}
Plus généralement que dans le paragraphe précédent, si $X_1,X_2,\ldots$ est une suite de variable aléatoires suivant la même loi (mêmes espérances $m$ et mêmes écarts-type $s$), alors leur somme $S_n=X_1+\cdots+X_n$, pour $n$ assez grand, suit approximativment une loi normale $\mathcal{N}\left(nm,\sqrt{n}s\right)$.\par
\rema de petits phénomènes hasardeux de même type, dans la nature, s'«ajoutent», et la contribution de chacun permet d'obtenir à grande échelle des distributions ressemblant à la courbe de la loi normale (appelée aussi courbe de Gauss).





\end{multicols}



\end{document}
