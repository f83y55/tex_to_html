\documentclass[a4paper,11pt]{article}
\setlength{\columnseprule}{1pt}
\usepackage[utf8]{inputenc}
\usepackage[francais]{babel}
\usepackage[a4paper]{geometry}
\geometry{verbose,tmargin=1cm,bmargin=1cm,lmargin=1cm,rmargin=1cm,headheight=5mm,headsep=5mm,footskip=5mm}

\usepackage[pdftex]{graphicx}
\usepackage{amsmath}
\usepackage{amssymb}
\usepackage{graphics}
\usepackage{latexsym}
\usepackage{eurosym}
\usepackage{multicol}
\usepackage{array}
\usepackage{variations}
\usepackage{cancel}
\usepackage{enumerate}
\usepackage{xcolor}
\usepackage{titlesec}


\definecolor{colortitle1}{RGB}{0,50,50}
\definecolor{colortitle2}{RGB}{120,0,0}

\definecolor{colorsection}{RGB}{120,0,0}
\titleformat{\section}
{\color{colorsection}\normalfont\Large\bfseries}
{\color{colorsection}\thesection}{1em}{}

\definecolor{colorsubsection}{RGB}{0,90,0}
\titleformat{\subsection}
{\color{colorsubsection}\normalfont\large\bfseries}
{\color{colorsubsection}\thesubsection}{1em}{}

\renewcommand{\FrenchLabelItem}{$\bullet$}

\font\ensemble=msbm10
\def\N{\hbox{\ensemble{N}}}
\def\Z{\hbox{\ensemble{Z}}}
\def\Q{\hbox{\ensemble{Q}}}
\def\R{\hbox{\ensemble{R}}}
\def\F{\hbox{\ensemble{F}}}
\def\D{\hbox{\ensemble{D}}}
\def\e{\textrm{e}}

\newcounter{numdefi} \setcounter{numdefi}{1}
\def\defi{\textbf{Définition \thenumdefi~: \addtocounter{numdefi}{1} }}
\newcounter{numprop} \setcounter{numprop}{1}
\def\prop{\textbf{Propriété \thenumprop~: \addtocounter{numprop}{1} }}
\newcounter{numexem} \setcounter{numexem}{1}
\def\exem{\textbf{Exemple \thenumexem~: \addtocounter{numexem}{1} }}
\newcounter{numlemm} \setcounter{numlemm}{1}
\def\lemm{\textbf{Lemme \thenumlemm~: \addtocounter{numlemm}{1} }}
\newcounter{numcoro} \setcounter{numcoro}{1}
\def\coro{\textbf{Corollaire \thenumcoro~: \addtocounter{numcoro}{1} }}
\newcounter{numrema} \setcounter{numrema}{1}
\def\rema{\textbf{Remarque \thenumrema~: \addtocounter{numrema}{1} }}
\newcounter{numdemo} \setcounter{numdemo}{1}
\def\demo{\textbf{Démonstration \thenumdemo~: \addtocounter{numdemo}{1} }}
\newcounter{nummeth} \setcounter{nummeth}{1}
\def\meth{{\color{colorsubsection}\textbf{Méthode \thenummeth~: \addtocounter{nummeth}{1} }}}
\newcounter{numexo} \setcounter{numexo}{1}
\definecolor{colorexo}{RGB}{70,0,90}
\def\exo{{\color{colorexo}\textbf{Exercice \thenumexo~: \addtocounter{numexo}{1} }}}



\parskip=2mm
\date{}
\author{}
\title{}
\begin{document}

\begin{center} \Huge{\color{colortitle1}\fbox{\textbf{DS1 -- Séries de Fourier}}}\normalsize \end{center}
\textbf{Prénom NOM :}\hrulefill
\section{Signaux à tracer \emph{(6\,pts)}}
\emph{Tracer, ci-dessous, sur au moins deux périodes, les signaux dont les caractéristiques sont données ci-après :}
\begin{multicols}{2}
\begin{enumerate}[a)]
\item $g$ est un signal pair et $T$-périodique défini sur $\left[ 0;\pi \right]$ par :
$$g(t)=\left|\begin{array}{l} \cos t  \textrm{~sur~} \left[ 0; \frac{\pi}{2}\right[  \\
			     0\textrm{~sur~} \left[ \frac{\pi}{2} ; \pi \right]  \\ 
				      	\end{array} \right.$$
\item  $h$ est un signal impair et $4$-périodique défini sur $\left[ 0; 2 \right]$ par :
$$h(t)=\left|\begin{array}{l}  1 \textrm{~sur~} \left[ 0 ; 1 \right[  \\
				2-t   \textrm{~sur~} \left[ 1 ; 2\right]  \\ 
				      	\end{array} \right.$$
\item  $h$ est un signal pair et $4$-périodique défini sur $\left[ 0; 2 \right]$ par :
$$u(t)=\left|\begin{array}{l}  -t \textrm{~sur~} \left[ 0 ; 1 \right[  \\
				-1   \textrm{~sur~} \left[ 1 ; 2\right]  \\ 
				      	\end{array} \right.$$
\item  $h$ est un signal impair et $4$-périodique défini sur $\left[ 0; 2 \right]$ par :
$$v(t)=\left|\begin{array}{l}  1-t \textrm{~sur~} \left[ 0 ; 1 \right[  \\
				0   \textrm{~sur~} \left[ 1 ; 2\right]  \\ 
				      	\end{array} \right.$$
\end{enumerate}
\end{multicols}
\begin{enumerate}[a)]
\item \includegraphics[width=18cm]{signalvide1.png}\\
\item \includegraphics[width=18cm]{signalvide2.png}\\
\item \includegraphics[width=18cm]{signalvide2.png}\\
\item \includegraphics[width=18cm]{signalvide2.png}\\
\end{enumerate}

\vspace{2cm}

\begin{flushright}
Tourner SVP $\rightarrow$
\end{flushright}

\newpage
\section{Problème \emph{(14\,pts)}}
On considère la fonction $f$, périodique de période $T$, dont une représentation graphique est donnée par la figure ci-dessous :
\begin{center}
\includegraphics[width=19cm]{fourier.png}
\end{center}
Le développement en série de Fourier de cette fonction $f$ est noté : $\displaystyle a_0+\sum_{a\geq1}\left[a_n\cos (n\omega t)+b_n\sin(n\omega t)\right]$
\begin{enumerate}
\item Cette question est un QCM. Pour chaque affirmation, une seule des propositions est exacte. \\ 
\emph{Cocher la case correspondant à la réponse exacte.}
\begin{multicols}{2}
	\begin{enumerate}[a)]
	\item La période $T$ de la fonction $f$ est :\\
		$\square$ $T=3$\\
		$\square$ $T=5$\\
		$\square$ $T=6$\\
		$\square$ $T=+\infty$\\
	\item La valeur moyenne $a_0$ de la fonction $f$ est :\\
		$\square$ $a_0=0,\!5$\\
		$\square$ $a_0=0,\!6$\\
		$\square$ $a_0=1$\\
		$\square$ $a_0=1,\!2$\\
		\vspace{3cm}
	\item Pour tout entier $n\geq1$, on a $b_n=0$ car :\\
		$\square$ $f$ est paire.\\
		$\square$ $f$ est impaire.\\
		$\square$ $f$ est périodique.\\
		$\square$ $f$ n'a pas d'harmoniques de rang impair.\\
	\item Pour tout entier $n\geq1$, on a\\ $a_n=\dfrac{4}{T}\int_0^\frac{T}{2}\,f(t)\cos(n\omega t)\,\textrm{d}t$ car :\\
		$\square$ $f$ est paire.\\
		$\square$ $f$ est impaire.\\
		$\square$ $f$ est une intégrale.\\
		$\square$ $f=\pi R^2$.\\
	\end{enumerate}
\end{multicols}
	\item On rappelle que $a_n=\dfrac{4}{T}\int_0^\frac{T}{2}\,f(t)\cos(n\omega t)\,\textrm{d}t$\\
	\emph{Démontrer, en calculant, que pour tout $n\geq1$, on a :}
	$$a_n=\dfrac{-4\sin\left(\dfrac{2\pi}{5}n\right)}{\pi n}$$
	\item On rappelle que la valeur efficace $f_{\textrm{eff}}$ de la fonction $f$ est donnée par : $f_{\textrm{eff}}^2=\dfrac{1}{T}\displaystyle\int_0^T \,\left[f(t)^2\right]\,\textrm{d}t$\\
	\emph{Démontrer que $f_{\textrm{eff}}^2=2,\!4$}\\

	\item On note $S_n$ pour $n\geq 1$ entier la somme : $S_n=a_0^2+\dfrac{1}{2}\displaystyle\sum_{k=1}^n a_k^2$\\
	Le tableau suivant contient des valeurs approchées à $10^{-2}$.\\
	\emph{Compléter le tableau suivant :}
$$\begin{array}{|l||c|c|c|c|c|}\hline
n	&1	&2	&3	&4	&~~5~~~  \\ \hline
a_n	&-1,\!21&-0,\!37&~0,\!25~~&~0,\!30~~&~~~0~~~~  \\ \hline
a_n^2	&1,\!46	&0,\!14	&~0,\!06~~&~0,\!09~~&~~~0~~~~  \\ \hline
%S_n	&2,\!16	&2,\!23	&2,\!26&2,\!31	&2,\!31 \\ \hline
S_n	&2,\!16	&	&2,\!26&	& \\ \hline
\end{array}$$
\item \emph{Déterminer le plus petit entier $n$ tel que $S_n\geq0,\!95f_{\textrm{eff}}^2$.}
\end{enumerate}







\end{document}
